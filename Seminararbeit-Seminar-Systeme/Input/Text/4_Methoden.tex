\section{\acs{ki} zur Bekämpfung von Lebensmittelverschwendung}
\acf{ki} repräsentiert die Nachahmung menschlicher Denkfähigkeiten, wie logisches Denken, Lernen, Planen und Kreativität, durch eine Maschine nachzuahmen \cite{EuropaischesParlament.2020}. Diese sich entwickelnde Technologie beeinflusst Regierungen sowie traditionelle politische und wirtschaftliche Ansätze, um weltweite Herausforderungen zu bewältigen. Die jüngsten Fortschritte in \ac{ki}-Technologien wie Deep Learning, Bilderkennung, maschinelles Lernen und Sprachverarbeitung verdeutlichen, dass diese Technologien weiterhin einen erheblichen Einfluss auf den Alltag haben werden \cite{Sharma.2021}.

Über die letzten Jahre hinweg hat der Lebensmittelsektor erhebliche Investitionen in die Verarbeitung von Lebensmitteln getätigt, wobei der Schwerpunkt verstärkt auf Lieferketten und Logistik liegt. Diese Entwicklung führt zu einer Veränderung der Konsummuster und ermöglicht die Prognose der Marktsituation, insbesondere für Produkte mit kürzerer Haltbarkeit. Obwohl die Einführung von \ac{ki} und Big Data noch nicht flächendeckend erfolgt ist, zeigt sie bereits Unterschiede in der Gewinnrealisierung im Vergleich zu herkömmlichen Methoden und Techniken.

\subsection{Milch- und Molkereiprodukte}
Bei der Handhabung und Produktion von Milch- und Molkereiprodukte entstehen hohe Anforderungen an Qualität. Aufgrund ihrer leicht verderblichen Eigenschaften entstehen zudem erhöhte Transportkosten. Forschungsstudien zeigen, dass die Vorhersage der Milchproduktionserträge möglich ist \cite{Ho.2015}. Genauere Vorhersagen über Produkte liefern Informationen über Mangel, Effizienz und die Gesundheit der Kühe. Moderne Technologien in der Milchbeschaffung, Abrechnung, Produktzusammensetzung, Verpackung, Lieferkettenintegration und Rückverfolgbarkeit ermöglichen eine präzise Datenimplementierung zur Reduzierung von Zeit und Kosten. Durch physiologische Faktoren wie Herzfrequenz und Körpertemperatur sowie verschiedene Umweltfaktoren können mit Hilfe von \acs{ki} gesammelt und ausgewertet werden und bei den anschließenden Ergebnissen mit berücksichtigt werden . So wurde beispielsweise Hitzestress als Hauptursache für den Rückgang der Milchproduktion identifiziert \cite{Sugiono.2017}. Dabei wurden verschiedene Parameter wie pH-Wert, Prozentsatz löslichen Stickstoffs sowie Bakterien-, Hefe- und Schimmelzählungen berücksichtigt \cite{Goyal.2013}. Das Modell erweist sich als effizient, zeitsparend und hilfreich für die Lebensmittelsicherheit der Verbraucher.

\subsection{Wurst- und Fleischindustrie}
Der Ablauf des Ausbeinen, Herauslösen der Knochen aus dem Fleisch, sowie Zerlegen von Schlachttieren verläuft in einigen Schlachthöfen bereits voll automatisiert \cite{Buckingham.1995}, auch die Weiterverarbeitungen werden schrittweise umgerüstet. Nur mit Hilfe der Bilderkennung und direkten Auswertung ist dies möglich. Damit können den Menschen  in der Fleischindustrie schwere arbeiten abgenommen werden. Ein weiterer Vorteil beruht darauf, dass durch präzises und schnelles arbeiten die Hygiene weiter verbessert werden kann. 

Die Fleischindustrie benötigt moderne Analysemethoden zur schnellen Quantifizierung von Indikatoren, um geeignete Verarbeitungsverfahren für ihre Rohmaterialien zu bestimmen und die verbleibende Haltbarkeit ihrer Produkte vorherzusagen. In den vergangenen Jahren wurden relevante Analysen- und Screening-Methoden für Fleisch mithilfe von \ac{hplc} sowie Gaschromatographie-Massenspektrometrie durchgeführt \cite{Kodogiannis.2014}. Es werden verschiedene Methoden basierend auf analytischen Instrumentaltechniken wie der \ac{ftir} erforscht. Diese Methode zielt darauf ab, durch die Stoffwechselaktivität von Mikroorganismen auf Fleisch verursachte biochemische Veränderungen und die Bildung von Stoffwechselprodukten zu erfassen, die als einzigartige "Signatur" dienen und somit Informationen über Art und Geschwindigkeit des Verderbens liefern \cite{Nychas.2008}.

\subsection{Bachwaren}
Die Qualität von Backwaren wird durch viele Parameter bestimmt. Während der Produktion lassen sich dabei mittels Dichte- und Strukturanalysen Rückschlüsse auf die Zwischen- und Endproduktqualität ziehen. Die Veränderung der Dichte und Struktur von Teigen und Massen beeinflusst somit sowohl die Verarbeitbarkeit als auch die Qualität des Endprodukts.


