\section{Aktueller Stand}
Der Lebensmitteleinkauf wird oft nicht von einer geplanten Liste oder zwingenden Notwendigkeiten bestimmt, sondern von spontanen Impulsen und verlockenden Angeboten. Die Folge: Viele Produkte landen im Wagen, die vor Ablauf ihres Mindesthaltbarkeitsdatums niemals verbraucht werden können oder verderben - vom Regal direkt in die Tonne. Diese nichtige Entsorgung von Lebensmitteln, bekannt als Food Waste, ist weltweit ein ernstzunehmendes Problem. Spezifischen Schätzungen zufolge werden in Deutschland jährlich etwa 11 Millionen Tonnen Lebensmittel weggeworfen  
(Stand 2015), etwa die Hälfte davon vermeidbar \cite{Schmidt.2019}. 

Lebensmittelabfälle zählen zu den Verlusten, die sich durch die gesamte Lieferkette ziehen, von der landwirtschaftlichen Produktion zur Verarbeitung in der Nahrungsmittelindustrie, vom Einzelhandel bis hin zum Endverbraucher. 

\begin{figure}[h]
    \centering
    \begin{tikzpicture}
        \pie[text=legend, rotate = 90, explode={0, 0, 0, 0, 0}, radius = 3.5, color = {HMrot!20!white, HMrot!40!white, HMrot!60!white, HMrot!80!white, HMrot!100!white,}]{2/Primärerzeugung, 7/Handel, 15/Verarbeitung und Herstellung, 17/Gaststätten und Verpflegungsdienstleistungen, 59/Private Haushalte}
    \end{tikzpicture}
    \caption[Lebensmittelabfälle in Deutschland im Jahr 2020]{Lebensmittelabfälle in Deutschland im Jahr 2020 \cite{StatistischesBundesamt.2022}}
    \label{fig:AnteilAbfälle}
\end{figure}
\FloatBarrier

Wie in Abbildung \ref{fig:AnteilAbfälle} zu sehen,  ist mit $59\%$ ($6,5\,\text{Mio. t}$) der größte Anteil der Lebensmittelabfällen in Deutschland den privaten Haushalten zuzuordnen, dies entspricht etwa $78\,\text{kg}$ pro Person und Jahr. Einen weiteren großen Anteil mit $17\%$ ($1,9\,\text{t}$) besitzt die Außer-Haus-Verpflegung. Auch entlang der Herstellung fällt viel Abfall an, wobei die Primärproduktion einen Anteil von $2\%$ ($0,2\,\text{Mio. t}$), die Verarbeitung $15\%$ ($1,6\,\text{Mio. t}$) und der Handel $7\%$ ($0,8\,\text{Mio. t}$) den geringeren Teil ausmachen. Es ist deutlich, dass die Lebensmittelabfallproblematik in verschiedenen Phasen der Lieferkette und insbesondere im häuslichen Umfeld signifikante Ausmaße annimmt.


Frische ist nicht ohne ihren Preis, insbesondere in Bezug auf die Haltbarkeit. Frische Lebensmittel sind reich an Nährstoffen, förderlich für die Gesundheit und überzeugen durch ihren Geschmack. Aufgrund ihrer geringen Behandlung oder Verarbeitung neigen frische Lebensmittel jedoch dazu, eine kurze Haltbarkeitsdauer zu haben. Hinzu kommt, dass viele Verbraucher frische Lebensmittel im Haushalt nicht optimal lagern, was die Haltbarkeit weiter verkürzt. Aufgrund dieser Umstände wird in deutschen Haushalten besonders häufig frisches Obst und Gemüse unnötig entsorgt, da es nach dem Kauf schlicht verdirbt. Gemäß einer Umfrage machen rund 34 Prozent der weggeworfenen Lebensmittel frisches Obst und Gemüse aus.

Angesichts von weltweit rund 800 Millionen Menschen, die Hunger leiden, und einer ausgeprägten sozialen Schere auf nationaler Ebene, wird die sorglose Verschwendung zu einer ethischen Angelegenheit. Für die Herstellung von Nahrungsmitteln werden kostbare Ressourcen aufgebraucht und durch Freisetzung von Treibhausgasen das Klima strapaziert - eine effizientere Verwendung der produzierten Lebensmittel könnte eine übermäßige Herstellung eindämmen und Ressourcen bewahren. Eine Reduktion von Lebensmittelabfällen wirkt sich demnach nicht nur positiv auf das eigene Portemonnaie aus, sondern ist zugleich eine bedeutende Maßnahme zum Schutz des Klimas.