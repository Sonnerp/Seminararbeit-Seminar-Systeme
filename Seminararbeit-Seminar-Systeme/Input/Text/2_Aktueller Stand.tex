\section{Projekte}
\subsection{PigVisio}
Das Projekt PigVisio, entwickelt von der Hochschule Kempten, konzentriert sich so beispielsweise auf die Organbegutachtung von Schlachtschweinen, bei der die Organe auf potenzielle Krankheiten untersucht werden. Auffälligkeiten führen dazu, dass die Organe nicht in den Lebensmittelkreislauf gelangen dürfen. Dieser Prozess ist entscheidend für die Lebensmittelsicherheit und liefert auch wichtige Erkenntnisse zum Tierwohl. Jährlich werden etwa 10\,Millionen von rund 52\,Millionen geschlachteten Schweinen mit Organbefunden identifiziert. Die erkannten Befunde werden an die jeweiligen Landwirte weitergegeben und dienen als wichtiges Werkzeug zur weiteren Verbesserung des Tierwohls in den Betrieben. Die Organbegutachtung wird derzeit manuell von Fachpersonal wie amtlichen Tierärzten durchgeführt, was einige Herausforderungen mit sich bringt. Das intelligente Kamerasystem ist darauf trainiert, Organe und die dazugehörigen Krankheiten zu identifizieren. Es analysiert die Organe direkt im Förderband nacheinander in Sekundenschnelle. Die durch \ac{ki} erkannten Befunde werden mittels eines Ampelsystems dem Fachpersonal leicht verständlich visuell angezeigt. Bei Auffälligkeiten wird eine gelbe oder rote Beleuchtung aktiviert, woraufhin das Fachpersonal entsprechende Maßnahmen einleitet. Die Lösung PigVisio ermöglicht so eine zuverlässige und objektive Organbefundung, standardisiert und beschleunigt den Qualitätssicherungsprozess und entlastet das Fachpersonal. Die automatische Dokumentation sorgt für Transparenz und stärkt das Vertrauen über die gesamte Wertschöpfungskette. \cite{Visionpier.}

\subsection{REIF}
Das Projekt REIF der Technischen Hochschule Augsburg in Zusammenarbeit mit 31 Projektpartner hat den Einsatz von \ac{ki} entlang der gesamten Wertschöpfungskette betrachtet, um die Verschwendung von Lebensmitteln deutlich zu reduzieren, insbesondere bei Fleisch- bzw. Wurstwaren, Milchprodukten und Backwaren.  
