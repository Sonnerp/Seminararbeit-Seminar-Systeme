%% Seiteneinstellung
\documentclass[a4paper,12pt]{scrartcl} %Dokument Klasse ohne chapter, wegen Header
\usepackage[left=2cm,right=2cm, top=3cm, bottom=3cm]{geometry} %Seitenränder
\linespread{1.5} %zeilenabstand
\setkomafont{disposition}{\bfseries}
%\usepackage{times}

%% Kopfzeile  & Fußzeile
\usepackage[headsepline,footsepline]{scrlayer-scrpage} %Paket für Kopf-und Fußzeile incl. Line
\usepackage{scrlfile}
\pagestyle{headings}
\clearpairofpagestyles %Alle Platzhalter werden geleert um zu befüllen 
\ihead{\headmark} %Kopfzeile links
\automark{section} %Kapitelüberschriften in der Kopfzeile
\ohead{\includegraphics[height = 0.8 cm]{Input/Bilder/Hochschule.png}} %Kopfzeile rechts
\ofoot{\pagemark} %Seitenanzahl Rechts einfügen

%% Allgemeine Pakete
\usepackage[utf8]{inputenc}
\usepackage[T1]{fontenc}
\usepackage[ngerman]{babel} %Rechtschreibung deutsch

\setcounter{secnumdepth}{5} %Nummerierung der Absätze
\setcounter{tocdepth}{5} %Wie viele Ebenen im Inhaltsverzeichnis stehen

\usepackage{eurosym} %Eurosymbol
\usepackage{lipsum} %Platzhalter-Text
\usepackage{xcolor} %Farben

%% Schriftart
\renewcommand{\rmdefault}{ptm} %Fließtext, phv für Helvetica, ptm für Times new Roman
\renewcommand{\sfdefault}{ptm} %Überschriften, Beschreibungen, phv für Helvetica

%Datum
\usepackage{advdate}
\newcommand{\yesterday}{{\AdvanceDate[-1]\today}}
\newcommand{\tomorrow}{{\AdvanceDate[1]\today}}

%% Absatz Einstellungen
\setlength{\parskip}{6pt}
\setlength{\parindent}{0cm}

%% Grafikpakete
\usepackage{graphicx} %erlaubt Grafiken einzubinden 
\usepackage{float} %erlaubt die Grafiken im Text zu fixieren
\usepackage{placeins} %\FloatBarrier verhindert, dass Text an Bildern/Tabellen vorbei kann.

%% Einbinden von PDFs
\usepackage{pdfpages}

%% Formeln/Mathematikumgebung
\usepackage{siunitx}
\usepackage{amsmath}

%%Tabellen%% 
\usepackage{booktabs} %Für horizontale Linien \toprule, \midrule, \bottomrule, \cmidrule
\usepackage{tabularx} %Für die Umgebung 
\usepackage{colortbl} %Für farbige Felder
\usepackage{array}
\newcolumntype{M}[1]{>{\centering\arraybackslash}m{#1}}
\usepackage[tableposition=top]{caption}
\usepackage{longtable}
\usepackage{rotating} %lässt Text rotieren
\newcommand\tabrotate[1]{\begin{turn}{90}\rlap{#1}\end{turn}}
\usepackage{multirow} %zeilenzusammenfassung
\usepackage{makecell} %Zeilenumbruch

%% Abkürzungsverzeichnis 
\usepackage[printonlyused]{acronym}

%%Literaturverzeichnis
\usepackage{csquotes}
\usepackage[hidelinks]{hyperref}
\usepackage[backend=biber,style=numeric, sorting=none]{biblatex}
\urlstyle{same}
\addbibresource{Input/Literatur/Literatur.bib} %Einbinden der passenden Datei
