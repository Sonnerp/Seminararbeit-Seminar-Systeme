%% Seiteneinstellung
\documentclass[a4paper,11pt]{scrartcl} %Dokument Klasse ohne chapter, wegen Header
\usepackage[left=2cm,right=2cm, top=3cm, bottom=3cm]{geometry} %Seitenränder
\linespread{1.5}
\setkomafont{disposition}{\bfseries}
\usepackage{times}

%% Kopfzeile  & Fußzeile
\usepackage[headsepline,footsepline]{scrlayer-scrpage} %Paket für Kopf-und Fußzeile incl. Line
\usepackage{scrlfile}
\pagestyle{headings}
\clearpairofpagestyles %Alle Platzhalter werden geleert um zu befüllen 
\ihead{\headmark} %Kopfzeile links
\automark{section} %Kapitelüberschriften in der Kopfzeile
\ohead{\includegraphics[height = 0.8 cm]{Input/Bilder/Hochschule.png}} %Kopfzeile rechts
\ofoot{\pagemark} %Seitenanzahl Rechts einfügen

%% Allgemeine Pakete
\usepackage[utf8]{inputenc}
\usepackage[T1]{fontenc}
\usepackage[ngerman]{babel} %Rechtschreibung deutsch

\setcounter{secnumdepth}{5} %Nummerierung der Absätze
\setcounter{tocdepth}{5}

\usepackage{eurosym} %Eurosymbol
\usepackage{lipsum} %Platzhalter-Text
\usepackage{xcolor} %Farben

%% Schriftart
\renewcommand{\rmdefault}{phv} %Fließtext, phv für Helvetica, ptm für Times new Roman
\renewcommand{\sfdefault}{phv} %Überschriften, Beschreibungen, phv für Helvetica

%% Absatz Einstellungen
\setlength{\parskip}{6pt}
\setlength{\parindent}{0cm}

%% Grafikpakete
\usepackage{graphicx} %erlaubt Grafiken einzubinden 
\usepackage{float} %erlaubt die Grafiken im Text zu fixieren

%% Einbinden von PDFs
\usepackage{pdfpages}

%% Formeln/Mathematikumgebung
%\usepackage{mathptmx}
%\usepackage{amsmath}
%\usepackage{amssymb}
%\usepackage{siunitx}

%% Abkürzungsverzeichnis 
\usepackage[printonlyused]{acronym}

%%Literaturverzeichnis
\usepackage{csquotes}
\usepackage[hidelinks]{hyperref}
\usepackage[backend=biber,defernumbers=false]{biblatex}
\addbibresource{Input/Literatur.bib} %Einbinden der passenden Datei


%% \FloatBarrier
\usepackage{placeins}


%%% Seiteneinstellung
\documentclass[a4paper,12pt]{scrartcl} %Dokument Klasse ohne chapter, wegen Header
\usepackage[left=2cm,right=2cm, top=3cm, bottom=3cm]{geometry} %Seitenränder
\linespread{1.5} %zeilenabstand
\setkomafont{disposition}{\bfseries}
%\usepackage{times}

%% Kopfzeile  & Fußzeile
\usepackage[headsepline,footsepline]{scrlayer-scrpage} %Paket für Kopf-und Fußzeile incl. Line
\usepackage{scrlfile}
\pagestyle{headings}
\clearpairofpagestyles %Alle Platzhalter werden geleert um zu befüllen 
\ihead{\headmark} %Kopfzeile links
\automark{section} %Kapitelüberschriften in der Kopfzeile
\ohead{\includegraphics[height = 0.8 cm]{Input/Bilder/Hochschule.png}} %Kopfzeile rechts
\ofoot{\pagemark} %Seitenanzahl Rechts einfügen

%% Allgemeine Pakete
\usepackage[utf8]{inputenc}
\usepackage[T1]{fontenc}
\usepackage[ngerman]{babel} %Rechtschreibung deutsch

\setcounter{secnumdepth}{5} %Nummerierung der Absätze
\setcounter{tocdepth}{5} %Wie viele Ebenen im Inhaltsverzeichnis stehen

\usepackage{eurosym} %Eurosymbol
\usepackage{lipsum} %Platzhalter-Text
\usepackage{xcolor} %Farben

%% Schriftart
\renewcommand{\rmdefault}{ptm} %Fließtext, phv für Helvetica, ptm für Times new Roman
\renewcommand{\sfdefault}{ptm} %Überschriften, Beschreibungen, phv für Helvetica

%Datum
\usepackage{advdate}
\newcommand{\yesterday}{{\AdvanceDate[-1]\today}}
\newcommand{\tomorrow}{{\AdvanceDate[1]\today}}

%% Absatz Einstellungen
\setlength{\parskip}{6pt}
\setlength{\parindent}{0cm}

%% Grafikpakete
\usepackage{graphicx} %erlaubt Grafiken einzubinden 
\usepackage{float} %erlaubt die Grafiken im Text zu fixieren
\usepackage{placeins} %\FloatBarrier verhindert, dass Text an Bildern/Tabellen vorbei kann.

%% Einbinden von PDFs
\usepackage{pdfpages}

%% Formeln/Mathematikumgebung
\usepackage{siunitx}
\usepackage{amsmath}

%%Tabellen%% 
\usepackage{booktabs} %Für horizontale Linien \toprule, \midrule, \bottomrule, \cmidrule
\usepackage{tabularx} %Für die Umgebung 
\usepackage{colortbl} %Für farbige Felder
\usepackage{array}
\newcolumntype{M}[1]{>{\centering\arraybackslash}m{#1}}
\usepackage[tableposition=top]{caption}
\usepackage{longtable}
\usepackage{rotating} %lässt Text rotieren
\newcommand\tabrotate[1]{\begin{turn}{90}\rlap{#1}\end{turn}}
\usepackage{multirow} %zeilenzusammenfassung
\usepackage{makecell} %Zeilenumbruch

%% Abkürzungsverzeichnis 
\usepackage[printonlyused]{acronym}

%%Literaturverzeichnis
\usepackage{csquotes}
\usepackage[hidelinks]{hyperref}
\usepackage[backend=biber,style=numeric, sorting=none]{biblatex}
\urlstyle{same}
\addbibresource{Input/Literatur/Literatur.bib} %Einbinden der passenden Datei


\begin{document}

\begin{titlepage}
    \raggedleft
    \includegraphics[width=0.65\textwidth]{Input/Bilder/HM-Logo-rot-Schriftzug.png}

    \centering
	\vspace{2cm}
    {\LARGE\bfseries Fakultät Elektrotechnik und Informationstechnik\\}
	
    \vfill
	{\large\bfseries Studienfach\\}
	{\LARGE\bfseries Seminar Systeme WS2023/24\\}
	
    \vfill
	{\large\bfseries Seminararbeit\\}
    {\LARGE\bfseries Einsatz von Künstlicher Intelligenz zur Abfallvermeidung in der Lebensmittelindustrie\\} 

    \raggedright
    \vfill
    {\large\bfseries Von: Patricia Sonner\\}
    {\large\bfseries Betreuer: Prof. Dr. Gregor Feiertag\\}
    {\large\bfseries Abgabetermin: 08.01.2024\\}
\end{titlepage}

\clearpage
\pagenumbering{Roman}

\addsec{Kurzfassung}
\lipsum[1]
\newpage

%Inhaltsverzeichnis
\tableofcontents
\newpage

%Abkürzungsverzeichnis
\addsec{Abkürzungsverzeichnis}

\begin{acronym}[LONGEST]
    \acro{akw}[AKW]{Atomkraftwerk}
\end{acronym}
\newpage

%Abbildungsverzeichnis
\addcontentsline{toc}{section}{Abbildungsverzeichnis}
\listoffigures
\newpage

%Tabellenverzeichnis
\addcontentsline{toc}{section}{Tabellenverzeichnis}
\listoftables
\newpage

%%Hauptteil
\clearpage
\pagenumbering{arabic}

\section{Einleitung}
Der Lebensmitteleinkauf wird oft nicht von einer geplanten Liste oder zwingenden Notwendigkeiten bestimmt, sondern von spontanen Impulsen und verlockenden Angeboten. Die Folge, viele Produkte landen im Wagen, die vor Ablauf ihres Mindesthaltbarkeitsdatums niemals verbraucht werden können oder verderben - vom Regal direkt in die Tonne. Diese nichtige Entsorgung von Lebensmitteln, auch bekannt als Food Waste, ist weltweit ein ernstzunehmendes Problem. Spezifischen Schätzungen zufolge werden in Deutschland jährlich etwa 11 Millionen Tonnen Lebensmittel weggeworfen, etwa die Hälfte davon vermeidbar \cite{Schmidt.2019}. 

Lebensmittelabfälle zählen zu den Verlusten, die sich durch die gesamte Lieferkette ziehen, von der landwirtschaftlichen Produktion zur Verarbeitung in der Nahrungsmittelindustrie, vom Einzelhandel bis hin zum Endverbraucher. 

\begin{figure}[h]
    \centering
    \begin{tikzpicture}
        \pie[text=legend, rotate = 0, explode={0, 0, 0, 0, 0}, radius = 3.5, color = {HMrot!20!white, HMrot!40!white, HMrot!60!white, HMrot!80!white, HMrot!100!white,}]{2/Primärerzeugung, 7/Handel, 15/Verarbeitung und Herstellung, 17/Gaststätten und Verpflegungsdienstleistungen, 59/Private Haushalte}
    \end{tikzpicture}
    \caption[Lebensmittelabfälle in Deutschland im Jahr 2020]{Lebensmittelabfälle in Deutschland im Jahr 2020 \cite{StatistischesBundesamt.2022}}
    \label{fig:AnteilAbfälle}
\end{figure}
\FloatBarrier

Wie in Abbildung \ref{fig:AnteilAbfälle} zu sehen,  ist mit $59\%$ ($6,5\,\text{Mio. t}$) der größte Anteil der Lebensmittelabfällen in Deutschland den privaten Haushalten zuzuordnen, dies entspricht etwa $78\,\text{kg}$ pro Person und Jahr. Einen weiteren großen Anteil mit $17\%$ ($1,9\,\text{Mio. t}$) besitzt die Außer-Haus-Verpflegung. Auch entlang der Herstellung fällt viel Abfall an, wobei die Primärproduktion einen Anteil von $2\%$ ($0,2\,\text{Mio. t}$), die Verarbeitung $15\%$ ($1,6\,\text{Mio. t}$) und der Handel $7\%$ ($0,8\,\text{Mio. t}$) den geringeren Teil ausmachen. Es ist deutlich, dass die Lebensmittelabfallproblematik in verschiedenen Phasen der Lieferkette und insbesondere im häuslichen Umfeld signifikante Ausmaße annimmt.
 
Die \ac{fao} hat in ihren statistischen Veröffentlichungen deutlich gemacht, dass etwa ein Drittel der Lebensmittel, die für den Verzehr produziert werden, verschwendet wird. Dies entspricht nahezu $1,3,\text{Mrd. t.}$ an Lebensmittelabfällen \cite{Cederberg.2011}. Darüber hinaus verursacht dies jährlich Kosten in Höhe von 936 Milliarden US-Dollar für die globale Wirtschaft \cite{Jamaludin.2022}.

Diese beeindruckenden Zahlen erscheint besonders besorgniserregend, wenn man die weltweit 735 Millionen Menschen in Betracht zieht, die unter Hunger leiden \cite{StatistischesBundesamt.2022b}. Diese Problematik hat nicht nur schwerwiegende Auswirkungen auf die Umwelt, sondern beeinflusst auch soziale und ökonomische Aspekte in beträchtlichem Maße.

Die Folgen erstrecken sich über verschiedene Ebenen, die Ressourcen, die für die Produktion dieser Lebensmittel eingesetzt werden, gehen verloren, was ökologische sowie ökonomische Konsequenzen nach sich zieht. Ebenso spielt eine optimierte Logistik sowie Lagerhaltung zum Handel sowie abschließend bis zum Endverbraucher eine entscheidende Rolle. Die Implementierung neuer Maßnahmen ist daher essenziell, um das Problem der Lebensmittelverschwendung anzugehen und einen positiven Einfluss auf die Ernährungssicherheit und die Reduzierung der Umweltbelastung auszuüben.

Die Reduzierung von Lebensmittelabfällen sollte in sämtlichen Phasen der Wertschöpfungskette Priorität haben. Obwohl viele Aspekte rund um die Produktion und den Handel von Lebensmitteln optimiert werden könnten, ist es angesichts des enormen Volumens und des täglichen Bedarfs kaum möglich, nicht verwendbare Lebensmittelabfälle vollständig zu vermeiden. Eine Verbesserung der Produktions- und Logistikprozesse, insbesondere in Verbindung mit einem bewussteren Umgang und einer höheren Wertschätzung von Lebensmitteln im Allgemeinen, kann jedoch zu Fortschritten führen und die Menge an verschwendeten Nahrungsmitteln reduzieren. Der Großteil der oft unnötigen Lebensmittelabfälle entsteht beim Endverbraucher.


\begin{figure}[h]
    \centering
    \begin{tikzpicture}[
        node distance = 2mm,
          start chain = going right,
         start/.style = {signal, draw=#1, fill=#1!100,
                         text width=30mm, minimum height=55mm,
                         signal pointer angle=150, on chain},
          cont/.style = {start=#1, signal from=west}
                         ]
        
        \node[start=HMrot!100!white] {\textbf{Landwirtschaft}\\
            \begin{itemize}
                \item Schwan\-kende Erträge
                \item Qualitäts\-schwan\-kungen
            \end{itemize}   };
        \node[cont=HMrot!80!white] {\textbf{Produktion}\\
            \begin{itemize}
                \item Verluste durch Überproduktion
                \item Falsche Bestellmengen
            \end{itemize}   };
        \node[cont=HMrot!60!white] {\textbf{Handel}\\
            \begin{itemize}
                \item Qualitäts\-ausschuss
                \item Falsche Bestellmengen
            \end{itemize}};     
        \node[cont=HMrot!40!white] {\textbf{Verbraucher}\\
            \begin{itemize}
                \item Schwan\-kende Nachfragen
                \item Falsche Lagerhaltung
            \end{itemize}};                     
    \end{tikzpicture}
    \caption[Lebensmittelverluste entlang der Wertschöpfungskette]{Lebensmittelverluste entlang der Wertschöpfungskette \cite{ProjektREIF.}}
    \label{fig:Verlustkette}
\end{figure}

Die Lebensmittelverarbeitung kann nun zyklisch erfolgen, dennoch entsteht eine komplexe Lieferkette, wie sie in Abbildung \ref{fig:Verlustkette} zu sehen ist und erfordert einen neuen Ansatz. Die Integration von \ac{ki}-Technologien bietet eine vielversprechende Perspektive, um innovative Lösungen für die Überwachung, Analyse und Optimierung der Lebensmittel\-produktions- und Lieferketten bereitzustellen. Es besteht das Potenzial, das Bewusstsein zu schärfen und weltweit nachhaltige Praktiken zu fördern, um die Herausforderungen der Lebensmittelverschwendung zu bewältigen. \ac{ki}-Technologien können effektive Überwachungsmechanismen etablieren, Engpässe identifizieren und Prozesse optimieren, um Ressourcen effizienter zu nutzen und Verschwendung zu reduzieren.

Ein besonderes Augenmerk soll in dieser Seminararbeit auf die Betrachtung von verderblichen Lebensmitteln im Herstellungsprozess geworfen werden. Dort treten häufig erhebliche Lebensmittelverluste auf, vor allem aufgrund der empfindlichen Verderblichkeit der Produkte. Darüber hinaus geht die Herstellung und Verarbeitung dieser Produkte mit einem erheblichen Aufwand einher, was zu hohen Produktionskosten führt.

\newpage

\section{Aktueller Stand}
Der Lebensmitteleinkauf wird oft nicht von einer geplanten Liste oder zwingenden Notwendigkeiten bestimmt, sondern von spontanen Impulsen und verlockenden Angeboten. Die Folge: Viele Produkte landen im Wagen, die vor Ablauf ihres Mindesthaltbarkeitsdatums niemals verbraucht werden können oder verderben - vom Regal direkt in die Tonne. Diese nichtige Entsorgung von Lebensmitteln, bekannt als Food Waste, ist weltweit ein ernstzunehmendes Problem. Spezifischen Schätzungen zufolge werden in Deutschland jährlich etwa 11 Millionen Tonnen Lebensmittel weggeworfen  
(Stand 2015), etwa die Hälfte davon vermeidbar \cite{Schmidt.2019}. 

Lebensmittelabfälle zählen zu den Verlusten, die sich durch die gesamte Lieferkette ziehen, von der landwirtschaftlichen Produktion zur Verarbeitung in der Nahrungsmittelindustrie, vom Einzelhandel bis hin zum Endverbraucher. 

\begin{figure}[h]
    \centering
    \begin{tikzpicture}
        \pie[text=legend, rotate = 90, explode={0, 0, 0, 0, 0}, radius = 3.5, color = {HMrot!20!white, HMrot!40!white, HMrot!60!white, HMrot!80!white, HMrot!100!white,}]{2/Primärerzeugung, 7/Handel, 15/Verarbeitung und Herstellung, 17/Gaststätten und Verpflegungsdienstleistungen, 59/Private Haushalte}
    \end{tikzpicture}
    \caption[Lebensmittelabfälle in Deutschland im Jahr 2020]{Lebensmittelabfälle in Deutschland im Jahr 2020 \cite{StatistischesBundesamt.2022}}
    \label{fig:AnteilAbfälle}
\end{figure}
\FloatBarrier

Wie in Abbildung \ref{fig:AnteilAbfälle} zu sehen,  ist mit $59\%$ ($6,5\,\text{Mio. t}$) der größte Anteil der Lebensmittelabfällen in Deutschland den privaten Haushalten zuzuordnen, dies entspricht etwa $78\,\text{kg}$ pro Person und Jahr. Einen weiteren großen Anteil mit $17\%$ ($1,9\,\text{t}$) besitzt die Außer-Haus-Verpflegung. Auch entlang der Herstellung fällt viel Abfall an, wobei die Primärproduktion einen Anteil von $2\%$ ($0,2\,\text{Mio. t}$), die Verarbeitung $15\%$ ($1,6\,\text{Mio. t}$) und der Handel $7\%$ ($0,8\,\text{Mio. t}$) den geringeren Teil ausmachen. Es ist deutlich, dass die Lebensmittelabfallproblematik in verschiedenen Phasen der Lieferkette und insbesondere im häuslichen Umfeld signifikante Ausmaße annimmt.


Frische ist nicht ohne ihren Preis, insbesondere in Bezug auf die Haltbarkeit. Frische Lebensmittel sind reich an Nährstoffen, förderlich für die Gesundheit und überzeugen durch ihren Geschmack. Aufgrund ihrer geringen Behandlung oder Verarbeitung neigen frische Lebensmittel jedoch dazu, eine kurze Haltbarkeitsdauer zu haben. Hinzu kommt, dass viele Verbraucher frische Lebensmittel im Haushalt nicht optimal lagern, was die Haltbarkeit weiter verkürzt. Aufgrund dieser Umstände wird in deutschen Haushalten besonders häufig frisches Obst und Gemüse unnötig entsorgt, da es nach dem Kauf schlicht verdirbt. Gemäß einer Umfrage machen rund 34 Prozent der weggeworfenen Lebensmittel frisches Obst und Gemüse aus.

Angesichts von weltweit rund 800 Millionen Menschen, die Hunger leiden, und einer ausgeprägten sozialen Schere auf nationaler Ebene, wird die sorglose Verschwendung zu einer ethischen Angelegenheit. Für die Herstellung von Nahrungsmitteln werden kostbare Ressourcen aufgebraucht und durch Freisetzung von Treibhausgasen das Klima strapaziert - eine effizientere Verwendung der produzierten Lebensmittel könnte eine übermäßige Herstellung eindämmen und Ressourcen bewahren. Eine Reduktion von Lebensmittelabfällen wirkt sich demnach nicht nur positiv auf das eigene Portemonnaie aus, sondern ist zugleich eine bedeutende Maßnahme zum Schutz des Klimas.
\newpage

\section{Wertschöpfungsketten}

\subsection{Molkereiindustrie}

\subsection{Fleisch- und Wurstwarenindustrie}

\subsection{Backwarenindustrie}


\newpage

\section{\acs{ki} zur Bekämpfung von Lebensmittelverschwendung}
\acf{ki} repräsentiert die Nachahmung menschlicher Denkfähigkeiten, wie logisches Denken, Lernen, Planen und Kreativität, durch eine Maschine nachzuahmen \cite{EuropaischesParlament.2020}. Diese sich entwickelnde Technologie beeinflusst Regierungen sowie traditionelle politische und wirtschaftliche Ansätze, um weltweite Herausforderungen zu bewältigen. Die jüngsten Fortschritte in \ac{ki}-Technologien wie Deep Learning, Bilderkennung, maschinelles Lernen und Sprachverarbeitung verdeutlichen, dass diese Technologien weiterhin einen erheblichen Einfluss auf den Alltag haben werden \cite{Sharma.2021}.

Über die letzten Jahre hinweg hat der Lebensmittelsektor erhebliche Investitionen in die Verarbeitung von Lebensmitteln getätigt, wobei der Schwerpunkt verstärkt auf Lieferketten und Logistik liegt. Diese Entwicklung führt zu einer Veränderung der Konsummuster und ermöglicht die Prognose der Marktsituation, insbesondere für Produkte mit kürzerer Haltbarkeit. Obwohl die Einführung von \ac{ki} und Big Data noch nicht flächendeckend erfolgt ist, zeigt sie bereits Unterschiede in der Gewinnrealisierung im Vergleich zu herkömmlichen Methoden und Techniken.

\subsection{Milch- und Molkereiprodukte}
Bei der Handhabung und Produktion von Milch- und Molkereiprodukte entstehen hohe Anforderungen an Qualität. Aufgrund ihrer leicht verderblichen Eigenschaften entstehen zudem erhöhte Transportkosten. Forschungsstudien zeigen, dass die Vorhersage der Milchproduktionserträge möglich ist \cite{Ho.2015}. Genauere Vorhersagen über Produkte liefern Informationen über Mangel, Effizienz und die Gesundheit der Kühe. Moderne Technologien in der Milchbeschaffung, Abrechnung, Produktzusammensetzung, Verpackung, Lieferkettenintegration und Rückverfolgbarkeit ermöglichen eine präzise Datenimplementierung zur Reduzierung von Zeit und Kosten. Durch physiologische Faktoren wie Herzfrequenz und Körpertemperatur sowie verschiedene Umweltfaktoren können mit Hilfe von \acs{ki} gesammelt und ausgewertet werden und bei den anschließenden Ergebnissen mit berücksichtigt werden . So wurde beispielsweise Hitzestress als Hauptursache für den Rückgang der Milchproduktion identifiziert \cite{Sugiono.2017}. Dabei wurden verschiedene Parameter wie pH-Wert, Prozentsatz löslichen Stickstoffs sowie Bakterien-, Hefe- und Schimmelzählungen berücksichtigt \cite{Goyal.2013}. Das Modell erweist sich als effizient, zeitsparend und hilfreich für die Lebensmittelsicherheit der Verbraucher.

\subsection{Wurst- und Fleischindustrie}
Der Ablauf des Ausbeinen, Herauslösen der Knochen aus dem Fleisch, sowie Zerlegen von Schlachttieren verläuft in einigen Schlachthöfen bereits voll automatisiert \cite{Buckingham.1995}, auch die Weiterverarbeitungen werden schrittweise umgerüstet. Nur mit Hilfe der Bilderkennung und direkten Auswertung ist dies möglich. Damit können den Menschen  in der Fleischindustrie schwere arbeiten abgenommen werden. Ein weiterer Vorteil beruht darauf, dass durch präzises und schnelles arbeiten die Hygiene weiter verbessert werden kann. 

Die Fleischindustrie benötigt moderne Analysemethoden zur schnellen Quantifizierung von Indikatoren, um geeignete Verarbeitungsverfahren für ihre Rohmaterialien zu bestimmen und die verbleibende Haltbarkeit ihrer Produkte vorherzusagen. In den vergangenen Jahren wurden relevante Analysen- und Screening-Methoden für Fleisch mithilfe von \ac{hplc} sowie Gaschromatographie-Massenspektrometrie durchgeführt \cite{Kodogiannis.2014}. Es werden verschiedene Methoden basierend auf analytischen Instrumentaltechniken wie der \ac{ftir} erforscht. Diese Methode zielt darauf ab, durch die Stoffwechselaktivität von Mikroorganismen auf Fleisch verursachte biochemische Veränderungen und die Bildung von Stoffwechselprodukten zu erfassen, die als einzigartige "Signatur" dienen und somit Informationen über Art und Geschwindigkeit des Verderbens liefern \cite{Nychas.2008}.

\subsection{Bachwaren}
Die Qualität von Backwaren wird durch viele Parameter bestimmt. Während der Produktion lassen sich dabei mittels Dichte- und Strukturanalysen Rückschlüsse auf die Zwischen- und Endproduktqualität ziehen. Die Veränderung der Dichte und Struktur von Teigen und Massen beeinflusst somit sowohl die Verarbeitbarkeit als auch die Qualität des Endprodukts.



\newpage

\section{Fazit}
Methoden der \ac{ki} sind bisher in den Bereichen der Milch, Fleisch und Backwaren wenig vertreten, besitzen aber Potenzial, die Verschwendung signifikant zu reduzieren. Nicht nur die Vermeidung von Ausschuss und Überproduktion, sondern auch die 
\newpage

%%Anhang
\clearpage 
\pagenumbering{Roman}

\input{Input/Text/6_Anhang.tex}
\newpage 

%Literaturverzeichnis nach Abbildungen, Bücher, Datenblatt, Normen, Internet
\addcontentsline{toc}{section}{Literaturverzeichnis}
\printbibliography[title=Literaturverzeichnis]
\newpage

\end{document}
